\include{includes}
\begin{document}


\titulo
\textbf{Propuestas de la Comunidad Universitaria y la Delegación de Alumnos de la ETSIT UPM para las candidaturas a director/a 2025}

\textbf{\textcolor{DATorange}}

\vspace{2cm}

\section*{\subtitulonaranja{Introducción}}

\cuerpo{
    Desde la Delegación de Alumnos de la Escuela Técnica Superior de Ingenieros de Telecomunicación (DAT) de la Universidad Politécnica de Madrid y con
    motivo de las elecciones a director/a de la ETSIT de este 2025, se ha realizado una recogida de
    propuestas a través de nuestra plataforma web \href{https://participa.dat.etsit.upm.es}{participa.dat.etsit.upm.es},
    por parte de todos los miembros de la comunidad universitaria en pro de la mejora del funcionamiento y el futuro de
    la Escuela.
}

\cuerpo{
    Tras dicha recogida de propuestas, se sometieron a una votación abierta a todos los miembros de nuestra comunidad universitaria. Estas se presentarán a todos los 
    candidatos, trasladando de esta manera el sentir de nuestros compañeros y compañeras en una lista de objetivos. Tras un proceso de negociación, los candidatos se podrán 
    comprometer a cumplir las demandas que consideren a lo largo de los próximos 6 años de legislatura en caso de salir elegidos.
}

\cuerpo{
    Posteriormente, se procederá a la publicación de este documento y la respuesta recibida en las redes sociales, página web y en la plataforma destinada a las elecciones 
    a director de la ETSIT del 2025, para uso y conocimiento de todos.
}

\cuerpo{
    A continuación, se presentan las propuestas apoyadas y sus matices o compromisos adquiridos por el candidato en caso de ser elegido/a director/a de la ETSIT.
}

\newpage

\section*{\subtitulonaranja{Propuestas y compromisos}}

%CONTENT%

\newpage

\section*{\subtitulonaranja{Conclusiones}}

\cuerpo{
    La participación activa de todos los colectivos que integran la Universidad Politécnica de Madrid en los procesos de toma de decisiones es fundamental de cara a la 
    mejora, desarrollo y evolución de la propia universidad. Para que esto sea posible, es imprescindible que se realice un proceso de diálogo entre todas las partes 
    implicadas, y que todas las demandas sean escuchadas y consensuadas
}

\cuerpo{
    Si se cumplen estas premisas, tarde o temprano, todos los participantes del proceso acaban sintiéndose parte del mismo, convirtiéndose de esta manera en un eslabón en 
    la cadena de valor de la Universidad, y fomentando su mejora, desarrollo e innovación. Es por ello que nosotros, como representantes de nuestros compañeros, 
    consideramos que este documento es nuestro aporte para alcanzar el objetivo de una Universidad Pública accesible y de calidad, siendo las siguientes medidas las que 
    consideramos fundamentales para poder alcanzar este objetivo.
}

\section*{\subtitulonaranja{Compromiso}}

\cuerpo{
    Mediante este documento, los candidatos a director/a de la ETSIT adquieren un compromiso con la comunidad universitaria de la UPM, comprometiéndose a cumplir con las propuestas seleccionadas en los términos indicados por ellos mismos en caso de resultar elegidos. 
}

\cuerpo{
    Una vez los candidatos hayan cumplimentado y firmado electrónicamente este documento, la delegación se compromete a su publicación a través de su página web y redes sociales, indicando cuáles de los candidatos que se presentan a director/a de esta Escuela han decidido hacer suyas las propuestas recogidas en este documento.
}

\begin{center}
    \begin{figure}[h]
        \centering
        \includegraphics[width=0.5\textwidth]{logoDAT.png}
    \end{figure}
    Firmado digitalmente por el candidato \\
    \textcolor{DATorange}{
      %CANDIDATE% \\
    }
\end{center}

\end{document}