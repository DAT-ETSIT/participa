%%%%%%%%%%%%%%%%%%%%%%%%%%%%%%%%%%%%%%%%%%%%%%%%%%%%%%%%%%%%%%%%%%%%%%%%%%%%%%
%
% Delegación de Alumnos de Telecomunicación
% PLANTILLA DE DOCUMENTOS EN LaTeX
%
% Esta plantilla consiste en la plantilla oficial de documentos de DAT,
% creada originalmente en LibreOffice y exportada a PDF, poniéndola como
% fondo en un documento de LaTeX estándar con algunas modificaciones extra
% (fuentes, márgenes, etc.).
%
%%%%%%%%%%%%%%%%%%%%%%%%%%%%%%%%%%%%%%%%%%%%%%%%%%%%%%%%%%%%%%%%%%%%%%%%%%%%%%

\documentclass[11pt]{article} % Documento de tipo "artículo", con fuente por defecto de 11pt.

\usepackage{amsmath}

%\usepackage{fontspec}
%\newfontfamily\comfortaa{Comfortaa}[
%    Extension=.ttf,%
%    Ligatures={TeX,Common},%
%    FontFace={l}{n}{*-Light},%
%    FontFace={l}{it}{Font=*-Light,FakeSlant=0.167},%
%    UprightFont={*-Regular},%
%    ItalicFont={*-Regular},%
%    ItalicFeatures={FakeSlant=0.167},%
%    FontFace={b}{n}{*-Bold},%
%    FontFace={b}{it}{Font=*-Bold,FakeSlant=0.167},%
%    BoldFont={*-Bold},%
%    BoldItalicFont={*-Bold},%
%    BoldItalicFeatures={FakeSlant=0.167}]
 

\usepackage[default]{comfortaa}
\usepackage[T1]{fontenc}      % Usa fuentes "ricas", que facilitan el copy-paste en la salida.
\usepackage[spanish]{babel}   % Traducción al español de meses, funciones matemáticas, etc.
\usepackage{enumitem}         % Ajuste de espacios entre elementos de listas.
\usepackage{graphicx}         % Permite la inserción de imágenes.
\usepackage{hyperref}         % Convierte las referencias en hiperenlaces y añade el comando \url.
\usepackage{wallpaper}        % Para la plantilla del fondo.
\usepackage{xcolor}           % Fuentes de colores.
\usepackage{fbb}              % Fuente para documentos.
\usepackage[                  % Márgenes personalizados.
  a4paper,
  lmargin=2cm,
  rmargin=2cm,
  tmargin=4.5cm,
  bmargin=3.5cm,
  headheight=3cm
]{geometry}
\usepackage{setspace}

\usepackage[]{hyphenat}   % Evita el corte de palabras cuando no caben en una unica linea.

% Fondo de la plantilla de DAT.
\ULCornerWallPaper{1}{background_DAT.pdf}

% Elimina el sangrado de cada párrafo.
\setlength{\parindent}{0cm}
% Espacio de 1 línea entre párrafos.
\setlength{\parskip}{\baselineskip}

% Elimina cabeceras/pies de página por defecto.
\pagestyle{empty}

% Elimina la separación adicional entre ítems de listas.
\setlist[enumerate]{itemsep=0mm}

% Definición de colores de la plantilla.
\definecolor{DATorange}{HTML}{EA470A}

% Para hacer pruebas
\usepackage{lipsum}

% Nuestras definiciones de tipos de párrafos
\newcommand{\titulo}{\centering\fontsize{20}{13.5}\textcolor{DATorange}\bfseries\comfortaa}

\newcommand{\subtitulo}{\raggedright\comfortaa\fontsize{16}{18.5}\bfseries}

\newcommand{\subtitulonaranja}{\raggedright\comfortaa\fontsize{16}{18.5}\bfseries\textcolor{DATorange}}

\newcommand{\subtitulocentro}{\centering\comfortaa\fontsize{16}{18.5}\bfseries}

\newcommand{\titulillo}{\raggedright\comfortaa\fontsize{13}{15}\bfseries}

\newcommand{\titulillonaranja}{\raggedright\comfortaa\fontsize{13}{15}\bfseries\textcolor{DATorange}}

\newcommand{\cuerpo}{\raggedright\comfortaa\fontsize{11}{13.5}\fontseries{l}\selectfont}

\newcommand{\resaltado}{\raggedright\comfortaa\fontsize{11}{13.5}\fontseries{l}\selectfont\textcolor{DATorange}}
